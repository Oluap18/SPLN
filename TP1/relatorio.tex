\documentclass{report}

\usepackage[latin1]{inputenc}
\usepackage[portuguese]{babel}
\usepackage{url}
\usepackage{enumerate}

\parindent=0pt
\parskip=2pt

\setlength{\oddsidemargin}{-1cm}
\setlength{\textwidth}{18cm}
\setlength{\headsep}{-1cm}
\setlength{\textheight}{23cm}

\title{Scripting no Processamento de Linguagens Naturais\\ (4� ano de Curso de MiEI)\\ \textbf{Trabalho Pr�tico 1}\\ Relat�rio de Desenvolvimento}
\author{Catarina Cardoso\\ (a75037) \and Paulo Guedes\\ (a74411) \and Pedro Cunha\\ (a73958)}
\date{27/10/2017}

\begin{document}

\maketitle

\tableofcontents

\chapter{Introdu��o} \label{intro}

\section*{Descri��o} \
No �mbito da unidade curricular de Scripting no Processamento de Linguagens Naturais, foi proposta a resolu��o de um trabalho pr�tico usando a linguagem Perl que identificasse os nomes pr�prios presentes num texto e os relacionasse entre si, originando assim uma tabela de relacionamentos. Nesta tabela que relaciona � poss�vel observar o n�mero de vezes que os nomes se encontram na vizinhan�a um do outro e, assim, concluir qual o grau de "simpatia" entre eles.

\section*{Estrutura do Relat�rio} \
Ap�s o cap�tulo introdut�rio, segue-se o cap�tulo~\ref{di} onde se exp�e detalhamente as decis�es de implementa��o tomadas aquando a realiza��o do projeto. Termina-se o relat�rio no cap�tulo~\ref{concl}, onde s�o especificados sintaticamente os passos tomados durante a realiza��o do trabalho pr�tico. No final, no ap�ndice~\ref{append}, podem-se encontrar os ficheiros usados na realiza��o do trabalho pr�tico.

\chapter{Decis�es e Implementa��o} \label{di}


\chapter{Conclus�o} \label{concl}

\label{append}
\appendix
\chapter{C�digo do Programa}

Apresenta-se a seguir o c�digo do programa que foi desenvolvido.
\begin{verbatim}

\end{verbatim}

\end{document} 